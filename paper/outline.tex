\documentclass[12pt]{article}
\usepackage{url}

\title{Outline flowCore paper}

\author{Florian Hahne*\\
  Nolwenn LeMeur*\\
  Byron Ellis\\
  Ryan Brinkman\\
  Perry Haaland\\
  Errol Strain\\
  Deepayan Sarkar\\
  Josef Spidlen\\
  Robert Gentleman
 }

\begin{document}
\maketitle

\vspace{2ex} The paper will deal with the capabilities offered by
flowCore for handling flow cytometry data in R. The main focus should
be on high-throughput and how our infrastructure can be used to deal
with such data. We don't want to write a user manual but rather
highlight the problems that arise from high throughput flow
applications and the solutions we offer. A strong point should be on
integration of our infrastructure in the flow world. flowCore is
intended to be a development platform for other scientists, we don't
want to provide full-fledged solutions but rather tools for others to
devise such solutions.


It would be nice to show these things on one data set in the form of a
use case, but that is open to discussion. The problem with such an
approach is that it often looks like a software manual which is not
what should be aiming for. 

The author list is also still up for discussion, listed here are all
the names that came up during the last telephone conference. NLM and
FH are supposed to share first authorship.

Target journal should be Cytometry, and there are two flavors:
Cytometry A deals with more general topics while Cytometry B is more
geared towards clinical applications. Florian suggests Cytometry A,
here is a link to the author instructions:
\url{http://www3.interscience.wiley.com/journal/33945/home/ForAuthors.html}.
They require the standard Materials and Methods, Results and
Discussion structure which might not be suited for our needs. FH
hasn't found out about alternative methods-type articles yet.  We can
suggest to the editor new format for software/methods oriented papaer?
(NLM)

Also it is not clear from their web page whether they support LaTeX
manuscript submissions but FH would strongly argue for using
LaTeX. They do accept pdf submission.

There is an svn repository set up at
\url{https://hedgehog.fhcrc.org/gentleman/bioconductor/trunk/madman/Rpacks/flowCore/paper}
which is also the repository for the flowCore source code. All
contributors should already have usernames and passwords, if not
please contact FH.



FH put the following very rough structure up for discussion:

\section{Introduction}

\subsection{High throughput flow}
A brief introduction to the flow technology and high throughput flow
in particular. 

\subsection{R and Bioconductor}
A couple of sentences what Bioconductor and R is all
about.

\subsection{Existing data standards and conventions}
How does the flow data world look like
FCS
Other flow software (what other software are handling multiple FCS files all together)
(NLM: should we mention prada and rflowcyt?) FH say: Maybe not, I
think it is time to move away from history and establish flowCore as
the one software for flow in Bioconductor. 

\section{Basic data structures}

\subsection{demands on flow data structures}
numeric raw data matrices
instrument-specific meta data
experiment meta data
memory issues, scale of data (tubes, plates)

\subsection{flowFrame}
to represent individual FCS files

\subsection{flowSet} 
experiments, multiple FCS files, plates


\section{Standard flow operations}
\subsection{transformations}
Different flavors, R's own transformation infrastructure allows for
arbitrary transforms

\subsection{gating}
gate definitions
capturing gating schemes, workflow
gate summary statistics


\section{Application, Outlook}
Challenges in high throughput cytometry
(QA, automated gating, class discovery) 

flowCore as toolbox

already in devel 'plug-in' packages (Visualization, QA, stats, ...)

\end{document}

